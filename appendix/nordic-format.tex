% changes:
% jan 12 2010 jh: corrected for new magnitudes
% mar  3 2010 jh: amplitude also in nm/s
% oct 24 2011 pv: type M-line
% oct 27 2011 jh: change definition of Q from probably volcanic to quake
\chapter{The Nordic format}
\label{app:nordic}
%Appendix 1 - The Nordic format 
\index{Nordic format}\index{Format, Nordic} 

Free columns are included for two purposes: 
\begin{enumerate}
\item To obtain a readable format 
\item To have some space for possible future extensions
\end{enumerate}


 Here are examples, top 3 lines for positioning only. 
\begin{verbatim}
         1         2         3         4         5         6         7 
1234567890123456789012345678901234567890123456789012345678901234567890123456789
.         .         .         .         .         .         .         .
-------------------------------------------------------------------------------

 1984 1022 2102 23.2 LE 69.330 27.440 11.0F NAO 34 5.2 3.8LNAO 4.0bPDE 3.2sISC1
 NORTHERN FINLAND                                                             3
 NRSA SZ IPN 1 D 2244 13.44 0345 1234.6 1.33 245.2 08.6 841022 120.2 3 5 12345 
 NRSA SZ ILG 1 D 2244 13.44 0345 1234.6 1.33 265.0 03.6 841022 120.2 3 5 12345 

 1985 510 21 5 16.1 LE 60.240   6.170 30.0F BER  6 2.3 3.8LNAO 4.0bPDE 3.2sISC 
                1.5     0.5     0.9    5.0             0.4                    5 
 8505210425.WNN                                                               6
 NORTHERN HORDALAND       F 3.5 61.22 0.5   5.33 0.8  23456 2   456 2 99 11BER1
 STAT SP IPHASW D HRMM SECON CODA AMPLIT PERI AZIMU VELO SNR                  7
 BER  SZ IPG  2 U 2105 25.41  200
 HYA  SZ ISG  1   2105 33.1
 ODD  SZ IP   3   2105 20.1   250
 ODD  SZ EPG      2105 22.9
 ODD  SZ  LG      2105 55.8 

-------------------------------------------------------------------------------
\end{verbatim}
Below are examples of how the last free columns of type 4 lines are used in the Nordic Databank in Helsinki and in Bergen: 
\begin{verbatim}
 1985 510 21 5 16.1 LE 60.240 6.170 30.0F BER  6 2.3 3.8LNAO 4.0bPDE 3.2sISC  1 
                1.5     0.5   0.9    5.0             0.4                      5
 8505210425.WNN                                                               6
 ACTION:UPD 93-07-09 09:40 OP:jens STATUS:               ID:19920101080359    I
 STAT SP IPHASW D HRMM SECON CODA AMPLIT PERI AZIMU VELO SNR AR TRES W DIS CAZ7
 NRSA SZ IPN  1 D 2105 13.44 0345 1234.6 1.33 245.2 08.6 5.5  2 -0.7 9 555 235
 BER  SZ IPG  2 U 2105 25.41  200
 HYA  SZ ISG  1   2105 33.1
 ODD  SZ IP   3   2105 20.1   250
 ODD  SZ EPG      2105 22.9
 ODD  SZ  LG      2105 55.8 
\end{verbatim}
Note in this example the fault plane solution line(F) and the HYP error line(E) 
\begin{verbatim}
 1993 1028 0800 26.4 L 57.518 7.119 18.8  BER  6  .6 2.6CBER                   1
 GAP=201        1.20    6.4   7.0    6.8   .3359E+01  -.2719E+00      .3054E+02E 
 93.2 74.8 -48.2 2 F
 ACTION:SPL 95-01-08 09:40 OP:jh   STATUS:               ID:19931028080019     I
 9310-28-0800-19S.NSN\_17                                                      6
 STAT SP IPHASW D HRMM SECON CODA AMPLIT PERI AZIMU VELO SNR AR TRES W DIS CAZ7
 BLS5 SZ EP     D 8  0 56.80  129                                -.110 216 349
 BLS5 SZ ESG      8  1 23.59                                     -.910 216 349
 BLS5 SZ EP       8  0 56.80  129                                -.110 216 349
 BLS5 SZ ESG      8  1 23.59                                     -.910 216 349 

Location parameters:
AR  : Azimuth residual when using azimuth information in locations
TRES: Travel time residual 

W   : Actual weight used for location ( inc. e.g. distance weight), i2
DIS : Epicentral distance in km
CAZ : Azimuth from event to station 

-----------------------------------------------------------------------------
Note: Type 1 line must be the first, all type 4 lines should be together and
the last line must be blank 
--------------------------------------------------------------------------------

\end{verbatim}
 FORMAT DESCRIPTION: 
\begin{verbatim}
Type 1 Line:

Columns Format Description                    Comments 
  1            Free 
  2- 5   I4    Year 
  6            Free 
  7- 8   I2    Month 
  9-10   I2    Day of Month
 11            Fix o. time                    Normally blank, an F fixes origin time 
 12-13   I2    Hour 
 14-15   I2    Minutes 
 16            Free 
 17-20   F4.1  Seconds 
 21            Location model indicator       Any character
 22      A1    Distance Indicator             L = Local, R = Regional, etc.
 23      A1    Event ID                       E = Explosion, etc.
                                              P = Probable explosion
                                              V = Volcanic 
                                              Q =  Confirmed earthquake
 24-30   F7.3  Latitude                       Degrees (+ N)
 31-38   F8.3  Longitude                      Degrees (+ E)
 39-43   F5.1  Depth                          Km 
 44      A1    Depth Indicator                F = Fixed, S = Starting value
 45      A1    Locating indicator             ----------------------------, * do not locate 
 46-48   A3    Hypocenter Reporting Agency
 49-51         Number of Stations Used 
 52-55         RMS of Time Residuals 
 56-59  F4.1   Magnitude No. 1
 60 A1         Type of Magnitude L=ML, b=mb, B=mB, s=Ms, S=MS, W=MW, 
                                 G=MbLg (not used by SEISAN), C=Mc
 61-63  A3     Magnitude Reporting Agency
 64-67  F4.1   Magnitude No. 2
 68 A1         Type of Magnitude
 69-71  A3     Magnitude Reporting Agency
 72-75  F4.1   Magnitude No. 3
 76 A1         Type of Magnitude
 77-79  A3     Magnitude Reporting Agency
 80 A1         Type of this line ("1"), can be blank if first
               line of event 

If more than 3 magnitudes need to be associated with the hypocenter in the first line, a
subsequent additional type one line can be written with the same year, month, day until event
ID and hypocenter agency. The magnitudes on this line will then be associated with the main
header line and there is then room for 6 magnitudes. 

Type 2 line (Macroseismic information) 

  1-5          Blank 
  6-20         a Any descriptive text
 21            Free 
 22    a1      Diastrophism code (PDE type)
                      F = Surface faulting 
                      U = Uplift or subsidence
                      D = Faulting and Uplift/Subsidence
 23    a1      Tsunami code (PDE type)
                      T = Tsunami generated 
                      Q = Possible tsunami 
 24    a1      Seiche code (PDE type)
                      S = Seiche 
                      Q = Possible seiche 
 25    a1      Cultural effects (PDE type)
                      C = Casualties reported 
                      D = Damage reported 
                      F = Earthquake was felt
                      H = Earthquake was heard
 26    a1      Unusual events (PDE type)
                      L = Liquefaction 
                      G = Geysir/fumerol 
                      S = Landslides/Avalanches 
                      B = Sand blows 
                      C = Cracking in the ground (not normal faulting).
                      V = Visual phenomena 
                      O = Olfactory phenomena
                      M = More than one of the above observed. 
 27            Free 
 28-29 i2      Max Intensity
 30    a1      Max Intensity qualifier
                      (+ or - indicating more precicely the intensity)
 31-32 a2      Intensity scale (ISC type defintions)
                      MM = Modified Mercalli    
                      RF = Rossi Forel 
                      CS = Mercalli - Cancani - Seberg
                      SK = Medevev - Sponheur - Karnik33 Free 
 34-39 f6.2    Macroseismic latitude (Decimal)
 40            Free 
 41-47 f7.2    Macroseismic longitude (Decimal)
 48            Free 
 49-51 f3.1    Macroseismic magnitude
 52    a1      Type of magnitudeI = Magnitude based on maximum Intensity.
                      A = Magnitude based on felt area.
                      R = Magnitude based on radius of felt area.
                      * = Magnitude calculated by use of special formulas
                          developed by some person for a certain area.
                          Further info should be given on line 3.
 53-56 f4.2    Logarithm (base 10) of radius of felt area.
 57-61 f5.2    Logarithm (base 10) of area (km**2) number 1 where
                      earthquake was felt exceeding a given intensity.
 62-63 i2      Intensity boardering the area number 1.
 64-68 f5.2    Logarithm (base 10) of area (km**2) number 2 where
                      earthquake was felt exceeding a given intensity.
 69-70 i2      Intensity boardering the area number 2.71 Free
 72    a1      Quality rank of the report (A, B, C, D) 73-75 a3 Reporting agency
 76-79         Free
 80    a1      Type of this line ("2") 

Type 3 Line (Optional): 

Columns Format Description 	Comments 

  1            Free 
  2-79 A       Text      	Anything
 80    A1      Type of this line ("3") 

 This type of line can be used to specify xnear, xfar and the starting depth for use with
HYPOCENTER. For example 

 XNEAR  200.0 XFAR  400.0 SDEP   15.0                                          3

   8-13  f6.1  Xnear 
  20-25  f6.1  Xfar 
  32-36  f5.1  Starting depth 

Type 4 line: 

Columns Format Description 	Comments 

  1 Free 
  2- 6 A5 Station Name 	Blank = End of readings = end of
event 
  7 A1 Instrument Type S = SP, I = IP, L = LP etc
  8 A1 Component 	Z, N, E ,T, R, 1, 2
  9 Free or weight, see note below 
 10 A1 Quality Indicator 	I, E, etc.
 11-14 A2 Phase ID 	PN, PG, LG, P, S, etc. **
 15 I1 Weighting Indicator (1-4) 0 or blank= full weight, 1=75%, 2=50%, 3=25%,
              4=0%, 9: no weight, use difference
             time (e.g. P-S).
 16 Free or flag A to indicate automartic pick, removed when picking
 17 A1 First Motion 	C, D
 18 Note: Currently 15 to 18 can also be used for phase assuming
            column 11-14 is not blank. See note ** below. 
 19-20 I2 Hour 	Hour can be up to 48 to
             indicate next day 

 21-22 I2 Minutes 
 23-28 F6.0 Seconds 
 29 Free 
 30-33 I4 Duration (to noise) 	Seconds 
 34-40 g7.1 Amplitude (Zero-Peak) in units of nm, nm/s, nm/s^2 or counts.
 41 Free 
 42-45 F4.0 Period 	Seconds 
 46 Free 
 47-51 F5.0 Direction of Approach 	Degrees
 52 Free 
 53-56 F4.0 Phase Velocity 	Km/second
 57-60 F4.0 Angle of incidence (was Signal to noise ratio before version 8.0)
 61-63 I3 Azimuth residual 
 64-68 F5.1 Travel time residual 
 69-70 I2 Weight
 71-75 F5.0 Epicentral distance(km)
 76 Free 
 77-79 I3 Azimuth at source 
 80 A1 	Type of this line ("4"), can be blank, which it is
               most often 

NB: Epicentral distance: Had format I5 before version 7.2. All old lines can be read with
format F5.0 with same results, but now distance can also be e.g. 1.23 km which cannot be read
by earlier versions. However, an UPDATE would fix that. 
\end{verbatim}
\index{Long phase names}
\index{Phase name}
\begin{verbatim}
  ** Long phase names: An 8 character phase can be used in column 11-18. There is then not
room for polarity information. The weight is then put into column 9. This format is recognized
by HYP and MULPLT. 

Type 4 cards should be followed by a Blank Card (Type 0) 

Type 5 line (optional): Error estimates of previous line, currently not used
                        by any SEISAN programs. 

Columns Format Description 	Comments 
  1 Free
  2-79 Error estimates in same format as previous line, normallytype 4
 80 A1 Type of this line ("5") 

Type 6 Line (Optional): 

Columns Format Description 	Comments 
  1 Free 2-79 A Name(s) of tracedata files80 A1 Type of this line ("6") 

Type 7 Line (Optional): 
  
Columns Format Description 	Comments 

  1 Free
  2-79 A Help lines to place the numbers in right positions
 80 A1 Type of this line ("7") 

Type E Line (Optional): Hyp error estimates 

Columns Format Description

  1 Free
  2 - 5 A4 The text GAP= 
  6 - 8 I3 Gap
 15-20 F6.2 Origin time error
 25-30 F6.1 Latitude (y) error
 31-32 Free
 33-38 F6.1 Longitude (x) error (km)
 39-43 F5.1 Depth (z) error (km)
 44-55 E12.4 Covariance (x,y) km*km
 56-67 E12.4 Covarience (x,z) km*km
 68-79 E14.4 Covariance (y,z) km*km 

\end{verbatim}
%
%Type F Line (Optional): Fault plane solution 
%
%Columns Format Description
%
%  1-30 3F10.0 Strike, dip and rake, Aki convention
% 31:36 I6 	Number of bad polarities
% 61:63 A3 	Agency code
% 71:76 A6 	Method or source of solution, seisan mames INVRAD or FOCMEC
% 78:78 A1 	Quality of solution, A (best), B C or D (worst), added manually
% 79:79 A1 	Blank: Prime solution, overwritten when focmec or invrad
%                makes a new solution, non blank: remain in file, cannot be
%                plotted
%                O: Remain in file and can be plotted 
%
\begin{verbatim}
Type F Line (Optional): Fault plane solution 

Columns Format Description

  1:30 3F10.0 Strike, dip and rake, Aki convention
 31:45 4F5.1  Error in strike dip and rake (HASH), error in fault plane and aux. plane (FPFIT) 
 46:50 F5.1   Fit error:  FPFIT and HASH (F-fit)
 51:55 F5.1   Station distribution ratio (FPFIT, HASH)
 56:60 F5.1   Amplitude ratio fit (HASH, FOCMEC)
 61:65 I2     Number of bad polarities (FOCMEC, PINV) 
 64.65 I2     Number of bad amplitude  ratios (FOCMEC)
 67:69 A3     Agency code
 71:77 A7     Program used
 78:78 A1     Quality of solution, A (best), B C or D (worst), added manually
 79:79 A1     Blank, can be used by user
 80:80 A1     F

Type H line, High accuracy hypoenter line 

Columns 

  1:55 As type 1 line
 16 Free
 17 Seconds, f6.3
 23 Free 
 24:32 Latitude, f9.5 
 33 Free 
 34:43 Longitude, f10.5
 44 Free 
 45:52 Depth, f8.3
 53 Free 
 54:59 RMS, f6.3
 60:79 Free
 80 H 

Type I Line, ID line 

Columns Format description1 Free 

  2:8 Help text for the action indicator
  9:11 	Last action done, so far defined SPL: Split
           REG: Register
           ARG: AUTO Register, AUTOREG
           UPD: Update
           UP : Update only from EEV
           REE: Register from EEV
           DUB: Duplicated event
           NEW: New event 
 12 Free 
 13:26 Date and time of last action 
 27 Free 
 28:30 Help text for operator
 36:42 Help text for status
 43:56 Status flags, not yet defined
 57 Free 
 58:60 Help text for ID
 61:74 ID, year to second 
 75 If d, this indicate that a new file id had to be created which was
        one or more seconds different from an existing ID to avoid overwrite. 
 76 Indicate if ID is locked. Blank means not locked, L means locked. 

Type M Line (Optional): Moment tensor solution 

Note: the type M lines are pairs of lines with one line that gives the hypocenter time,
and one line that gives the moment tensor values:

The first moment tensor line:
Columns Format Description
  1:1          Free 
  2: 5   I4    Year 
  7: 8   I2    Month 
  9:10   I2    Day of Month
 12:13   I2    Hour 
 14:15   I2    Minutes 
 17:20   F4.1  Seconds 
 24:30   F7.3  Latitude                       Degrees (+ N)
 31:38   F8.3  Longitude                      Degrees (+ E)
 39:43   F5.1  Depth                          Km 
 46:48   A3    Reporting Agency
 56:59  F4.1   Magnitude 
 60     A1     Type of Magnitude L=ML, b=mb, B=mB, s=Ms, S=MS, W=MW, 
 61:63  A3     Magnitude Reporting Agency
 71:77  A7     Method used
 78:78  A1     Quality of solution, A (best), B C or D (worst), added manually
 79:79  A1     Blank, can be used by user
 80:A1         M

The second moment tensor line:
Columns Format Description
  1:1          Free 
  2:3   A2     MT
  4:9   F6.3   Mrr or Mzz [Nm]
 11:16  F6.3   Mtt or Mxx [Nm]
 18:23  F6.3   Mpp or Myy [Nm]
 25:30  F6.3   Mrt or Mzx [Nm]
 32:37  F6.3   Mrp or Mzy [Nm]
 39:44  F6.3   Mtp or Mxy [Nm]
 46:48  A3     Reporting Agency
 49:49  A1     MT coordinate system (S=spherical, C=Cartesian)
 50:51  i2     Exponental
 53:62  G6.3   Scalar Moment [Nm]
 71:77  A7     Method used
 78:78  A1     Quality of solution, A (best), B C or D (worst), added manually
 79:79  A1     Blank, can be used by user
 80:80  A1     M

Type E13 and EC3 line, explosion information 

Example  

1980 0124 0927 CHARGE(T): 0.5 E13  LE Haakonsvern,  HAA underwater explosion  E13 EC3  

Information on explsion site, time and agency, same format as a type 1 line, no magnitudesused, last 3 columns must be E13 
EC3 Information on charge and site 

Columns 
  2:11 Info text 
 11:12 Blank 
 13:22 Charge in tons, f10.3
 23:77 Any information, a
 78:80 EC3 

Type MACRO3 line: File name of macroseismic observations in ISO directory 

Example: 

1980-03-14-0456-05.MACRO MACRO3 

An example of the file is: 

Sunnfjord 1980 314 456 5 GMT 1980 314 556 5 Local time
 Comment 
60.500 5.270 1.0 EMS 5088 MJOELKERAAEN 
60.560 5.260 1.0 EMS 5100 ISDALSTOE 
60.570 5.050 1.0 EMS 5112 ROSSLAND 

1.  Line 
Location, GMT time, Local time. Format a30,i4,1x,2i2,1x,2i2,1x,i2,'
GMT',1x,i4,1x,2i2,1x,2i2,1x,i2,1x,'Local time'
2.  Line Comments 
3.  Line Observations: Latitude, Longitude,intensity, code for scale, postal code or similar,
location,Format 2f10.4,f5.1,1x,a3,1x,a10,2x,a. Note the postal code is an ascii string and
left justified (a10). 

Type 3 line giving xnear/xfar 

Definition of xnear and xfar to be used with HYPOCENTER. 

Example
XNEAR 1000.0 XFAR 2000.0 3 
Columns
  8-13: xnear value
 20-25: xfar value 


\end{verbatim}
