\section{ML inversion, MAG2}
\index{MAG2}
\label{MAG2}

MAG2 is a program to invert for the local magnitude scale ML. 
The difference to the inversion done in MAG is that MAG2 inverts 
amplitudes from all events simultaneously for the scale and 
station corrections. The program can invert for different scale 
parameters depending on selected distance ranges. The reason for 
this is that it is known that the geometrical spreading is not 
the same for example between Pg and Pn. Some authors have suggested 
distance dependent scales, but most commonly a single scale is used 
for all distances for simplicity.

The general ML scale is given by 

\begin{displaymath}
ML = log_{10} A + a log_{10}(R) + b R + S + c
\end{displaymath}

where we measure the displacement amplitude A in nm, R is the hypocentral 
distance in km, S is the station correction of the individual stations, 
and c is a constant added to make the scale comparable to other places at a 
reference distance. The station corrections add up to 0. The region 
dependant parameters in the scale are a, accounting for geometrical 
spreading, and b, accounting for attenuation. The part 
$(a log_{10}(R) + b R + c)$ is commonly written as $(-log_{10} A0)$. 

The program applies singular value decomposition using 
the Numerical Recipe \citep{press2003} routines to invert 
the observations for a, b and S. It then computes the parameter 
c based on the reference given through distance, amplitude and 
magnitude. This allows to calibrate scales between different 
regions so that they are the same at the reference distance. 
Commonly c is set such that 480 nm amplitude at 17 km gives ML=2 
(this is equivalent to 10 mm on a Wood-Anderson seismograph 
giving ML=3 at 17 km \citep{hutton1987}. The inversion 
can be setup to invert for the geometrical spreading  term a 
in the scale for up to three distance ranges. However, a single 
attenuation term b is used.

As input the program requires a parameter file mag2.par in the 
working directory,and a standard station file e.g. STATION0.HYP. 
Then the user only has to enter the input file of events in Nordic 
format. An example file, mag2.par, and an input file, mag2nor.cat, 
with events from Norway are in given in DAT., 

The parameter file has the following settings given by keywords (any order):

INVERSION TYPE (f10.1) - 1. = singular value decomposition (no other choice yet)\newline
DISTANCES (2f10.1) - distance range in km for observations to use\newline
MINIMUM NUMBER OF OBS/EVEN (f10.1) - only events with more or equal number 
of observations are used\newline
MIN DISTANCERANGE RATIO (f10.1) - minimum range required computed as ratio 
of distances defined by DISTANCES\newline
ORIENTATION (f10.1) - use of components: 0. = horizontal and vertical, 
1. = horizontal only, 2. = vertical only\newline
SYNTHETIC (f10.1) - set to 1. for synthetic test, scale defined by FIX SCALE A and FIX SCALE B; 0. for inversion of data\newline
NOISE (f10.1) - ratio of amplitude to be added as noise to synthetic test\newline
FIX SCALE A (3f10.1) - set the fixed parameter a in scale, possible for the 
three distance ranges given by SCALE DISTANCE\newline
FIX SCALE B (3f10.1) - set the fixed b parameter in scale, possible for the 
three distance ranges given by SCALE DISTANCE\newline
FIX SITE (f10.1) - set to 1. to not invert for station corrections; 0. for 
default inversion for station corrections\newline
REFERENCE DISTANCE, REFERENCE AMPLITUDE, REFERENCE MAGNITUDE 
(all f10.1) - setup of the reference, used to calculate parameter c, 
give amplitude as Wood Anderson amplitude in mm\newline
SCALE DISTANCE (2f10.1) - give up to two distances which give the 
transition between the possibly three distance dependent scales; blank 
or numbers larger than maximum distance will give only one scale\newline
IGNORE COMP (a4)  - give component not to be used\newline
IGNORE STAT (a5) - give station not to be used\newline


The program produces a number of output files:

mag2\_amp\_dis.out - amplitude versus distance for each event\newline
mag2\_amp\_obs.out - list of observed amplitudes\newline
mag2\_events\_read.out - listing of events that were read in\newline
mag2\_events\_used.out - events used in Nordic format\newline
mag2\_evxy.out - coordinates of events used\newline
mag2.out - general output file, lists data used and computed scale\newline
mag2\_paths.out - event-station path coordinates\newline
mag2\_station\_hyp.out - hyp station file with scale and station corrections\newline
mag2\_stat\_list.out - simple output of stations used\newline
mag2\_statxy.out - coordinates of stations used\newline

The output file mag2.out will give some details on the input data, as number of 
stations, events and observations. It reports the reference used to fix the scale at 
the reference distance. Next it gives the scale, consisting of three parts if inversion 
is done for all possible segments. This will be given by a1, a2, a3, while b will be 
assumed to be the same. First the scale is presented to include the reference distance, 
second it is shown without the reference distance included with the scale. Then comes a 
section with the stations and the respective site terms, and finally the list of events 
with the source term inverted for.

The output file mag2.out for the example in the DAT directory should look like this:

\begin{verbatim}
ML inversion output

 SVD inversion

 Total number of events:           69
 Total number of stations:           23
 Total number of observations:          600

 Reference distance  =    100.0000    
 Reference amplitude =    1.000000    
 Reference magnitude =    3.000000    

 Ml = log A + a log(dist/refdist) + b (dist-refdist) + c + S 
     a1=  0.84717 +/-  0.39844
     a2=  0.00000 +/-  0.00100
     a3=  0.00000 +/-  0.00100
     b =  0.00061 +/-  0.00136
     c =  0.31807 +/-  0.00000

 Ml = log A + a log(dist) + b (dist) + c + S 
    a1 =  0.84717 +/-  0.39844
    a2 =  0.00000 +/-  0.00100
    a3 =  0.00000 +/-  0.00100                                                                    
     b =  0.00061 +/-  0.00136
    c1 = -1.43679
    c2 =  0.25756
    c3 =  0.25756

Station #    1 STAV  -0.120  +/- 0.2612   58.935    5.702
Station #    2 BLS5  -0.044  +/- 0.2204   59.423    6.456
Station #    3 ODD1  -0.085  +/- 0.2125   59.911    6.627
Station #    4 EGD    0.043  +/- 0.2221   60.270    5.223
...
Station #   21 MOR8  -0.018  +/- 1.3061   66.285   14.732
Station #   22 ESK    0.851  +/- 1.1257   55.317   -3.205
Station #   23 AKN    0.171  +/- 0.3797   62.178    6.997
 Average site term:   0.00
  Event #     1 2002051922484590  ML =  1.95 +/- 0.438
  Event #     2 2002052614481700  ML =  1.68 +/- 0.577
...
\end{verbatim}

