% feb 14 11 jh: jh_change: add sample_read_arc
% jan 25 12 jh: add more sample programs
% dec 28 12 jh: change to sample_read_wav
%
\chapter{Programming in SEISAN and list of files in SEISAN distribution}
\label{chap:programming}

This chapter gives a bit more technical details of SEISAN starting 
with a short programmers guide with description of sample and test programs. 
\index{Test programs} 

\section{Programmers guide and some test programs}

SEISAN is conglomerate of programs and subroutines and it can be 
difficult to find out which routines to use and how to start a new 
SEISAN program. The most common method is to use an existing program 
and modify it. The intention with this section is to make it easier 
by providing a few sample programs which then can be modified to 
do specific tasks. 
The compilation of existing SEISAN programs has been described in 
section \ref{sect:compiling} and details of the commands are found in the Makefiles. 
In this distribution, sample programs have been included, which 
each illustrate the used of some SEISAN features. All  programs 
are included in the Makefiles and can therefore be compiled directly, 
modified and recompiled. 

Reading and writing S-files 

A basic operation is to be able to read and write S-files, since 
all parameters are contained in the S-files. Starting with version 
7.2, a new library (\texttt{rea.for}) and include block (\texttt{rea.inc} 
for definition 
of variables) has been included to make it easier to read and 
write data into S-files. Earlier, S-files were only read and written 
as text strings and individual parameters were then read/written to 
the text strings. Now the new routines do it all. These routines are 
now used in a few programs, but will be included whenever a program 
is substantially modified. The sample program is called 
\texttt{sample\_read\_write\_s.for}. The program illustrates how to read all 
parameters in an S-file, make modifications and write out the file 
again. The program can be useful, if the user needs a program where 
special parameters are needed for a particular analysis or for output 
in another format. 
\index{Read S-file}\index{Write S-file}\index{S-file, parameters} 

Reading %\textcolor{red}{lo-change: 
and writing 
waveform files 

In SEISAN, waveform files can be in SEISAN, SAC, MiniSeed/SEED, Guralp  or 
GSE format. SEISAN format is slightly different depending on which 
compute platform it is written and byte swapping has to be done in 
some cases. In order to automatically handle the reading of waveform 
files, irrespective of format and computer platform, a set of standard 
routines are used (waveform.for) and an include block where all 
parameters and data end up (\texttt{waveform.inc}). 
\index{Waveform.inc}\index{Waveform.for}\index{SAMPLE\_READ\_WAV} 
The sample %\textcolor{red}{lo-change: 
reading 
program 
is called \texttt{sample\_read\_wav.for}.There is a similar program sample\_write\_wav for writing SEISAN waveform files.\index{sample\_write\_wav} 

The program illustrates how to read many waveform files belonging 
to one event as if it was one file, irrespective of format. It also 
demonstrates how to read just one waveform file. There is an output file which gives a listing of all different channels found in all the files read. Thsi listing is in the format used for defining channels in an achive. \index{Archive, channel specification} There is no detail 
on how to write a SEISAN binary file in this program, but some info 
is given under the format description in Appendix 
\ref{app:seisan-format} 
and the program \texttt{tsig.for} described below illustrates a simple write. 
The sample program sample\_read\_cont illustrates how to 
extract out a time segment of the continuous data base. The program 
also shows how to write a Seisan file with all headers. The program 
is started from the command prompt: 

sample\_read\_cont start\_time interval 

where start time is yyyymmddhhmmss and interval is interval in minutes. 

There is a similar program for reading data from archives, sample\_read\_arc. \index{SAMPLE\_READ\_ARC}

sample\_read\_write\_seed interval

Thsi program can read and write seed files using Chad and Rubens routines. it works independently of SEISAN subroutines.

\index {SAMPLE\_READ\_WRITE\_SEED}
\index{SAMPLE\_READ\_CONT} There is also a routine in Java available 
to read all SEISAN binary formats. The program is called SFORMAT 
\index{SFORMAT}\index{Java} (written by T. Utheim). Similarly there 
is a sample program to read all SEISAN binary formats in Perl 
(written by Angel Rodriguez). The program is called 
\texttt{seibinasc.pl} and you need a Perl interpreter to run it. 
Before starting the program, a DIRF must first be made of waveform 
input files. The output is identical to a SEISAN ASCII file as 
made by SEIASC. \index{Perl}\index{SEIBINASC} 

%\textcolor{red}{lo-change: 
The sample write program is called
\texttt{sample\_write\_wav.for}. It is a simple example 
of writing a straight line. The output formay is SEISAN.

Graphics in SEISAN 

SEISAN uses a set of graphics routines, which are identical in call 
on all 3 platforms . These routines then call low level routines 
which are platform dependent (X on Unix and Windows calls on PC). 
The programmer only has to use the high level routines. The routines 
also generate a PostScript output if a given parameter is set. 
The program is called \texttt{sample\_grapichs.for}. 
The program illustrates how to initiate graphics, make a few simple 
calls, get up and use the mouse and make a hard copy file. Most of 
the general graphics routines are located in file \texttt{seiplot.for} and 
common variables in \texttt{seiplot.inc}. The program can be 
useful for testing functionality of the mouse. 
\index{Graphics}\index{Test mouse}\index{Mouse, test}
\index{Test graphics}\index{SAMPLE\_GRAPHICS} 

Program LSQ is a simple example of how to make xy-graphics. It 
also shows how to make the output files for gmtxy. In order to find 
more info (apart from manual in INF) on gmtxy, see file 
\texttt{gmt.for} in LIB and \texttt{gmt\_xy\_par} 
in INF. \index{LSQ}\index{GMTXY} 

Program to make test signals, TSIG:
\index{Test signals, make}\index{TSIG}\index{Brune spectrum}

It is often useful to be able to work with controlled waveform data 
so a program making test signals is included. The program makes 
several traces, all with same length and sample rate and trace 1 
is the sum of all traces. For each trace selected, the parameters 
selected are: Frequency, amplitude (remember this is integer numbers 
in file so use at least 1000), phase, delay (delay time when the 
signal appears on trace relative to start of trace, the data before 
is zero) and damping. The damping is used to simulate seismometer 
damping or simple a damped signal and has a similar physical meaning 
as the seismometer damping constant, but period is not recalculated 
to simulate changing period with damping. Zero damping is no damping. 
\index{Signal, damped}\index{Damped signal} \newline
An additional trace can be made with a Brune displacement pulse 
generated with parameters corner frequency (f0), Q and kappa (see 
MULPLT) and travel time. Travel time is used for Q-correction and 
also places the pulse at travel time distance from the origin (start 
of trace), so length of trace must be longer than travel time. If 
zero q and kappa, no attenuation is used. The program also write 
an S-file with relevant parameters. The program illustrates a 
simple writing of a SEISAN waveform file.  

Java programs in SEISAN 

The Java programs are each given as a Jar-file,
% like \texttt{jseisan.jar}, 
the jar-files are located in the PRO directory. The jar-file contains 
all the Java source code, the Java classes and the project file so 
a user can decompress the jar file, change the script and make a 
new version of the program. The programs are started using a script 
file in the COM directory and no classpath has to be set, when SEISAN 
has been correctly installed. \index{Java} 

\section{CONTENTS OF PRO, LIB, INC, INF, COM, DAT, SUP, ISO and PIC DIRECTORIES}

The PRO, LIB, INC and COM directories contain software, the DAT 
directory parameter files for operating the SEISAN system and INF 
contains documentation and manuals. All files are listed and explained 
in the file \texttt{seisan.all} in the INF directory. 
The ISO directory contains macroseismic information. 

The program CHECK \index{Check}can check if a distribution is 
complete. Run CHECK and use option `basic'. The content of the 
distribution is compared to the \texttt{seisan.all} file in the INF directory. 

The \texttt{seisan.all} file also list programs no longer 
compiled, but with source code included in case there should 
be a future need for these programs 

