\begin{boxedverbatim}
HYP
Arrival time data input, select one:

  SEISAN database or                              : RETURN
  Alternative database, give 1-5 letter code      :
  Local index file, name must start with index or : 
  Local database, write ,, or                     :
  File name for one file in NORDIC format         :

                	Your answer here  determines  the input                 
source. A return means that you work directly on the BER database. A 1-5 letter 
code gives name of database, e.g. NAO. An index file or the name of a readings 
file  is used when you want to work on specific subsets.
                   Local database is S-files in local directory.

 Start Time           (YYYYMMDDHHMMSS) : 199012
 End Time, RETURN is to end of month   : 19901205 
          		Standard formatted time input.

 Interactive operation (N/Y=return)
                	If N, whole time interval or file is located, one line output pr event.  


#  1   1992 12 3 0137 40.3 NPHS=   12  T Q L #XXX  
#  2   1992 12 3 0237 43.3 NPHS=   14  T Q L #XXX  l   ! now locate 
here comes location, see HYP manual***************************** 
#  2   1992 12 3 0237 43.3 NPHS=   14  T Q L #XXX  q   ! stop 


 PRINT OUTPUT IN FILE print.out
 CAT-FILE IN FILE hyp.out
 Summary file in hypsum.out
 
\end{boxedverbatim}
