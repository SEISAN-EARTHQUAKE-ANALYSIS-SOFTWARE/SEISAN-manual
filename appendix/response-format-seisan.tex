\chapter{Response file formats used in SEISAN}
\label{app:response}
% "Appendix 3 - Response file formats used in SEISAN } 
% jan 23 2012 jh : add sac format

SEISAN can use either SEISAN response file format or GSE response file format.  The detailed SEISAN format is given here as well as the parts of the GSE format used in SEISAN. 

SEISAN response file format 

There are three SEISAN response formats. The instrument response can either be presented as (1) instruments constants, (2) pairs of frequency amplitude and phase or as (3) poles and zeros. 

\verbatiminput{appendix/seisan-resp.par}

Example of SEISAN FAP response file 

\verbatiminput{appendix/seisan-resp.fap}

Example of SEISAN PAZ response file using the same constants as above 

\verbatiminput{appendix/seisan-resp.paz}

GSE response file format 

Below is an example of a GSE response file generated by RESP for the same parameters as used for the SEISAN file above. The GSE response format is rather complex, and can contain parts that will not be understood by SEISAN. 

Below follows an example of the GSE response using FAP. The first line gives station and sensor type. The numbers following are gain in nm/c (0.15) at reference period (1. seconds), sample rate 
(20) and date. The following line (FAP2) gives a gain factor (1.) and the output units (V for Volts). Finally follows the frequency, gain and phase triplets. 

\verbatiminput{appendix/seisan-resp.gse.fap}

In the poles and zero representation, the file looks like: 

\verbatiminput{appendix/seisan-resp.gse.paz}

The first line is the same as before. The PAZ2 lines has the following meaning: Normalization constant of seismometer and filter (0.26e-5), number of poles and zeroes (2 and 3), and the type of response (Laplace transform). The DIG2 line has the gain of amplifier and AD converter combined (0.419e6 c/V) and sample rate. 

In the simplest case, the response is given by the PAZ and a scaling factor. It is common (like in SEED) to have two scaling constants, one that normalizes the PAZ to amplitude 1 at a calibration period, and another constant that gives the amplitudes in the physical units. This is NOT the case with the GSE2 format. The GSE2 response for PAZ normally contains at least two parts, the CAL2 line and a PAZ2 line. The scaling factor should scale the PAZ to output/input units, NOT normalize. In the CAL2 line, the system sensitivity at a calibration period is given in units input/output, but is generally not needed. The total response is given by the PAZ, multiplied with the PAZ2 scaling factor, or the product of several stages. 

This is how SEISAN reads the response, however, if it finds that the PAZ2 gives normalized values at the calibration period, the response is multiplied with the sensitivity given in the CAL2 line (this is done because such GSE files have been seen). 

FIR filters can be specified in GSE as an additional stage and can be written out by the RESP program. An example is given below. The FIR filter coefficients are required to completely describe the instrumentation. However, they are not used in SEISAN. 

FIR2 3 0.10E+01 3 0.030 A 180 \newline
0.18067658E-06 0.88848173E-06 0.24195760E-05  0.37699938E-05 0.32037496E-06 \newline
 -0.18581748E-04 -0.69605186E-04 -0.16990304E-03 -0.32499805E-03 -0.51408820E-03 \newline
 -0.68242848E-03 -0.75194239E-03 -0.65436028E-03 -0.37627667E-03  0.94138085E-05 \newline
0.35409257E-03 0.49653835E-03 0.35531260E-03 -0.29224902E-05 -0.37382543E-03 \newline
...
\index{GSE response file format} 

%jh

SAC format

Below is an exampel of the SAC response file format. Note that if zeros are zreo, there is no need to write them. The response must be in displacment with unit meters.

ZEROS 3
POLES 4
-0.0123  0.0123
-0.0123  -0.0123
-39.1800  49.1200
-39.1800  -49.1200
CONSTANT 3.832338e+12

\index {SAC response file format}

Note that the response file names must follow SEISAN response file file names  with SAC at the end like e.g. KEV\_\_B\_\_Z.2000-10-10-0000\_SAC. 
