%\chapter{SEISAN TRAINING COURSE} 
\chapter{SEISAN Training Course} 
\label{chap:training}

The document `Computer exercises in processing earthquake data using 
SEISAN and introduction to SEISAN' which is a tutorial for new users 
as well as experienced users, is included in the distribution. The 
testdata used in the exercises need to be installed, see %section 3
chapter \ref{chap:installation}. 
Going through the exercises of the tutorial might be the best way 
to learn SEISAN. The document is given as PDF file 
(\texttt{seitrain.pdf}) 
in the INF directory. 

The main goal of the introductory training course is to become familiar 
with the database program EEV, the plotting program MULPLT and the 
location program HYP. Of course additional reading of relevant sections 
in this manual is required. The basic exercises can be completed within 
one or two days, while the advanced exercises take more time. 

Content of the exercises 

\begin{enumerate}
\item[1] SEISAN basic exercises 
\item[2] Phase reading 
\item[3] Response files and seismic formats 
\item[4] Signal processing 
\item[5] Earthquake location 
\item[6] Magnitude 
\item[7] Focal mechanism 
\item[8] Spectral analysis and Q 
\item[9] Operation and earthquake statistics 
\item[10] Array analysis 
\item[11] Analysis of a data set 
\item[12] Data manipulation and import and export of data 
\end{enumerate}

