
\section{Merge events near in time ASSOCI}

\index{Merge events near in time} \index{ASSOCI}

The program will check if two events are close together in time and merge the events if requested. This is partly an alternative to use append in EEV. The program asks for maximum time difference between events to associate. The user will then be asked if events should be physically associated or not. The program is useful when merging a large number of events. The program has two alternatives for merging: 

\begin{enumerate}
\item
Merge events in same data base: One event is compared to the next event in the same data base. If they are close enough in time, the two events are merged and the program moves on to the next event. If 3 events are close in time, only the 2 first are merged. In order to also merge the third, the program has to be run again. 
\item
Merge events from a file into the data base: This option makes it possible to merge from another data base (use SELECT or COLLECT to create a file) without first completely mixing the two. The event from the file will be merged with as many files from the data base as fit the time difference criteria. So e.g. 2 events from the data base can both get the same event from the file included. At the end of the run, two files are output (associ\_rest.out associ\_merg.out) with events which were not merged and merged respectively. These can then be put into another data base with split, if desired. This function can also be used to separate the input file in two files.\index{Associ\_rest.out}

Note: When merging within one data base, the first event will get the next one merged into it. If merging from file into a data base, the event in the data base will always be the first and keep the main header. This thus a safe method when you want to keep the main header uncheged in the data base. 
\end{enumerate}

