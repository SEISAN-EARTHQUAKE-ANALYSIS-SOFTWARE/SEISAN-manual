%
% changes:
%
% mar 3 2010 jh: add guteneberg and richter
% jan 252012 jh: add dreger
%
\chapter{References}

Anderson, D. (1982). Robust earthquake location using M-estimates. PEPI, 30, 119-130. 
Banfill, R. (1996). PC-SUDS Utilities. A collection of tools for routine processing of seismic data stored in the seismic unified data system for DOS (PC-SUDS), Version 2.5, Small Systems Support. 
Boore,D.M. (1983). Stochastic simulation of high frequency ground motions based on seismological models of the radiated spectra. Bulletin of the Seismological Society of America, 73, 1865�1884. 
Boore,D.M. (1989). Quantitative ground motion estimates. In: Earthquake Hazards and the  Design of Constructed Facilities in the Eastern United States. K.H.Jacob and C.J.Turkstra (Eds.), Annals of the New York Academy of Sciences, 558, 81-94. 
Bouchon, M. (1981). A simple method for calculating Green's functions for elastic layered  media. Bull. Seism. Soc. Am. 71, 959-972. 
Brune,J.N. (1970). Tectonic stress and spectra of seismic shear waves. Journal of Geophysical Research, 75, 4997-5009. 
Capon, J., High-resolution frequency-wavenumber spectrum analysis. Proc. IEEE 57, 1408-1418, 1960. 
Chiu, J, B. L. Isacs and R. K. Cardwell (1986). Studies of crustal converted waves using short-period seismograms recorded in the Vanatu Island arc, Bull. Seism. Soc. Am. 76,177-190. 
Chapman, C.H., 1978. A new method for computing synthetic seismograms, Geophys.  
J. R. astr. Soc., 54, 481-518. Chapman, C.H. and Orcutt, J.A., 1985. The computation of body wave synthetic seismograms  in laterally homogeneous media, Reviews of Geophysics, 23, 105-163. Chapman, C.H., Chu Jen-Yi, and Lyness, D.G., 1988. The WKBJ seismogram algorithm, in:  D.J. Doornbos (ed.), Seismological algorithms, Academic Press, London, pp47-74. Dey-Sarkar, S.K. and Chapman, C.H. (1978). A simple method for computation of body-wave 
seismograms, Bull. Seismo. Soc Am., 68, 1577-1593. 
Draper, N.R. and Smith, H. (1966). Applied regression analysis, John Wiley and Sons, New York. 
Dreger, D. S (2003). TDMT_INV: Time domain seismic moment inversion. In International handbook of Earthquake and Engineering Seismology, ed Lee et al., Volume B 85.11, Boston and Amsterdal; Academic Press.
Ebel, J. E. and K. P. Bonjer (1990). Moment tensor inveriosn of small earthquakes in southwestern germany for fault plane solution. Geophys. J. Int. 101, 133-146. 
Goldstein, P. (1999). SAC user�s manual, Lawrence Livermore Laboratory, University of California. 
GSETT-3, (1997). Provisional GSE 2.1, Message Formats \& Protocols, Operations Annex 3. 
Gutenberg, B., and Richter, C. F. (1956). Magnitude and energy of earthquakes. Annali di Geofisica, 9, 1, 1-15.
Havskov, J and L Ottem�ller (2008). Processing earthquake data. Book in preparation, preliminary version at SEISAN web site fall 2008. 
Havskov, J and G. Alguacil (2004). Instrumentation in earthquake seismology. Springer 358 pp. 
Havskov, J, S. Malone, D McCloug and R. Crosson (1989). Coda Q for the state of Washington. Bull. Seism. Soc. Am.,79, 1024-1038. 
Herrmann, R. B. (1985). An extension of random vibration theory estimates of strong ground  motion to large distances, Bull. Seism. Soc. Am.,75, 1447-1453. 
Herrmann, R., B. and A. Kijko (1983). Modelling some empirical vertical component Lg relations, Bull. Seism. Soc. Am. 73,157-171. 
Herrmann, R. B. (1996). Computer programs in seismology. Manual, Saint Louis University. 
Hutton, L. K. and D. Boore (1987). The Ml scale in Southern California. Bull. Seism. Soc.  Am. 77,2074�2094. 
IRIS Consortium(1993). Standard for the Exchange of Earthquake Data � Reference Manual, 2nd Edition. 
Kanamori, H (1977). The energy release in great earthquakes. Journal of Geophysical Research, 82, 
1981-1987. Kissling, E., W.L. Ellsworth, D. Eberhart-Phillips and U. Kradolfer, (1994). Initial reference model in local earthquake tomography, Journal of Geophysical Research, Vol. 99, No. B10, 19 635-19 
646. Kissling, E., U. Kradolfer and H. Maurer, (1995). Program VELEST USERS GUIDE - Short  Intrduction. Klein, F. W. (1984) Users guide to HYPOINVERSE, a program for Vax and PC350 computers to solve for 
earthquake locations. USGS open file report 84-000. 
Kv�rna, T. and D.J. Doornbos, An integrated approach to slowness analysis with arrays and tree�component stations, in: NORSAR Semiannual Technucal Summary, 1 October 1985 - 31 March 1986, Scientific Report No. 2-85/86, NORSAR, Kjeller, Norway, 1986. 
Lee, W. H. K., R. E. Bennett and L. Meagher, 1972. A method for estimating magnitude of local earthquakes from signal duration, U.S.G.S Open file report. 
Lee, W.H.K. and Lahr, J.C. 1975. HYPO71 (revised): a computer program for determining hypocenter, magnitude and first motion pattern of local earthquakes. Open-file report, U.S. Geological Survey, 75-311. 
Lienert, B.R.E.,E. Berg and L. N. Frazer (1986). Hypocenter: An earthquake location method using centered, scaled, and adaptively least squares, BSSA, Vol  76, 771-783. 
Lienert, B. R. E. (1991). Report on modifications made to Hypocenter. Institute of Solid Earth  Physics, University of Bergen. 
Lienert, B.R.E and J. Havskov (1995). A computer program for locating earthquakes both locally and globally, Seismological Research Letters, 66, 26-36. 
McGuire,R.K. (1976). EQRISK. Evaluation of earthquake risk to site. United States Department of the Interior, Geological Survey, Open File Report 76-67, 90p. 
Menke, W., Holmes, R. C. and Xie, J. (2006). On the Nonuniqueness of the Coupled Origin Time�Velocity Tomography Problem, BSSA, 96, 1131-1139. 
Nakamura, Y. (1989). A method for dynamic characteristics estimation of subsurface using microtremors on the ground surface. Q. Rep. Railway Tech. Res. Inst., 30, 1989. 
Roberts, R.G., Christoffersson, A., and Cassidy, F., 1989. Real time events detection, phase  identification and source location estimation using single station component seismic data and a small PC, Geophysical Journal, 97, 471-480. 
Ordaz,M. (1991). CRISIS. Brief description of program CRISIS. Institute of Solid Earth Physics, University of Bergen, Norway, Internal Report, 16p. 
Ottem�ller, L. (1995). Explosion filtering for Scandinavia, Norwegian National Seismic Network technical report \# 2, IFJF, University of Bergen. 209 pp. 
Ottem�ller, L. and J. Havskov (1999). SeisNet: A General Purpose Virtual Seismic Network. SRL 70, 5, 522-528. 
Ottem�ller, L., N.M. Shapiro, S.K. Singh and J.F. Pacheco (2002). Lateral variation of Lg wave propagation in southern Mexico, J. Geophys. Res. 107. 
Ottem�ller, L. and J. Havskov (2003). Moment Magnitude Determination for Local and Regional Earthquakes Based on Source Spectra, Bull. Seism. Soc. Am., 93, 203-214. 
Peterson, J. (1993). Observation and modeling of seismic background noise, U.S. Geol., Survey Open-File report 93-322, 95p. 
Pujol, J. Software for joint hypocentral determination (2003). In "International Handbook of Earthquake and Engineering Seismology", Part B, Vol. 81, W.H.K. Lee, P. Jennings, H. Kanamori, C. Kisslinger, Eds. 
Ruud, B.O., E.S. Husebye, S.F. Ingate and A. Christoffersen (1988). Event location at any distance using seismic data from a single, three-component station. Bull. Seism. Soc. Am. 78, 308-325. Ruud, B.O., and Husebye, E.S., (1992). A new three-component detector and automatic single station bulletin production, Bull. Seism. Soc. Am., 82, 221-237. Scherbaum, F. (1996). Of Poles and Zeros � Fundamentals of Digital Seismology, Kluwer Academic Publishers. Singh, S.K., Apsel,R.J., Fried, J. and Brune,J.N. (1982). Spectral attenuation of SH-waves along the Imperial fault. Bulletin of the Seismological Society of America. 72, 2003-2016. 
Snoke, J. A., J. W. Munsey, A. G. Teague and G. A. Bollinger (1984). A program for focal  mechanism determination by combined use of polarity and SV-P amplitude ratio data. Earth quake notes, 55, p15. 
Waldhauser F. and W.L. Ellsworth (2000). A double-difference earthquake location algorithm: Method and application to the northern Hayward fault, Bull. Seism. Soc. Am., 90, 1353-1368.  
Waldhauser, F., hypoDD: A computer program to compute double-difference earthquake locations (2001). USGS Open File Rep., 01-113. 
Veith, K. F. and G. E. Clawson (1972). Magnitude from short period P-wave data. Bull. Seism. Soc. Am. 62, 435-440. 
